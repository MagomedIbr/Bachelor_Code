
\begin{abstractgerman}
Das Ziel dieser Bachelorarbeit ist es durch bioelektrische Signale der Gesichtsmuskulatur, den Sprachmodus sowie die Identität des Sprechers vorherzusagen. Dafür werden verschiedene Klassifikatoren in der Scikit-Learn Umgebung und der UKA-Korpus der CSL Bremen genutzt.

\paragraph{}
Durch Oberflächen-EMG wird die elektrische Muskelaktivität
gemessen und digitalisiert. Diese Daten werden genutzt, um verschiedene Klassifikatoren zu trainieren und zu testen. Von den getesteten Klassifikatoren hat sich der LDA Klassifikator als der beste herausgestellt.

\paragraph{}
Es wurden Cross-Session sowie Cross-User Experimente für den Sprachmodus durchgeführt, bei denen sich Cross-Session als besser erwiesen hat. Am Ende wurde durch Frequenzbänder kontrolliert, ob der Fokus auf einen einzelnen Muskel im Gesicht zu besseren Ergebnissen führen kann. Das konnte in dieser Arbeit nicht bestätigt werden.

\paragraph{}
Insgesamt konnte eine Genauigkeit für die Sprecher-Erkennung mit über 90 Prozent und eine Genauigkeit von höchstens 59 Prozent für die Sprachmodus-Erkennung erreicht werden. Dies bestätigt, dass die Sprecher-Erkennung sehr gut klassifiziert werden kann. Der Sprachmodus hat eine niedrigere Genauigkeit erreicht, zeigt aber trotzdem, dass es möglich ist diesen vorherzusagen. Das deutet, dass die Erkennung von Unterschieden in der Sprachproduktion erkennbar ist.
Die Genauigkeit ist nämlich deutlich über der Chance Wahrscheinlichkeit von 33 Prozent.


\clearpage
\end{abstractgerman}

